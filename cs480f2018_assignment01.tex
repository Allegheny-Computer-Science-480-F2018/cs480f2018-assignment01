\documentclass[11pt]{article}


\newcommand{\assignmentduedate}{13 September}
\newcommand{\assignmentassignedate}{ 6 September}
\newcommand{\assignmentnumber}{One}

\newcommand{\labyear}{2018}
\newcommand{\labtime}{2:30 pm}

\newcommand{\assigneddate}{Assigned:  \assignmentassignedate, \labyear{} at \labtime{}}
\newcommand{\duedate}{Due:  \assignmentduedate, \labyear{} at \labtime{}}


\usepackage{listings}
\lstset{
  basicstyle=\small\ttfamily,
  columns=flexible,
  breaklines=true
}

\usepackage{hyperref}
\hypersetup{
    colorlinks=true,
    linkcolor=blue,
    filecolor=magenta,      
    urlcolor=magenta,
}

\usepackage{fancyhdr}

\usepackage[margin=1in]{geometry}
\usepackage{fancyhdr}

\pagestyle{fancy}

\fancyhf{}
\rhead{Computer Science 480}
\lhead{ Assignment \assignmentnumber{}}
\rfoot{Page \thepage}
\lfoot{\duedate}

\usepackage{titlesec}
\titlespacing\section{0pt}{6pt plus 4pt minus 2pt}{4pt plus 2pt minus 2pt}

\newcommand{\labtitle}[1]
{
  \begin{center}
    \begin{center}
      \bf
      CMPSC 480 \\ Software Innovation I\\
      Fall 2018\\
      \medskip
    \end{center}
    \bf
    #1
  \end{center}
}

\begin{document}

\thispagestyle{empty}

\labtitle{Assignment \assignmentnumber{} \\ \assigneddate{} \\ \duedate{}}

\section*{Objectives}

To learn how to appropriately use social media for networking and how to create a social media marketing plan for yourself. To identify your past experiences, current skills, and to articulate a summary of your experiences in the field of computer science. To generate an informative and a professional LinkedIn profile and to use it to create connections that grow your professional network.  To learn how  to effectively navigate
the LinkedIn user interface and to assess how LinkedIn can be used for professional networking activities. To research different computer science areas Allegheny 
alumni are working in, to connect with professionals in the field, to explore employers, and to  join groups to network with professionals in the field of computer science.

\section*{Reading Assignment}

To do well on this assignment, you
should first read a blog titled \href{https://blog.bufferapp.com/social-media-marketing-plan}{How to Create a Social Media Marketing Plan From Scratch}. Next, you should read the article on media marketing strategies for  scientists, called \href{http://www.sciencemag.org/features/2014/02/scientists-guide-social-media}{A Scientist's Guide To Social Media} and and article on \href{https://www.businessnewsdaily.com/6758-social-media-mistakes.html}{Social Media Mistakes to Avoid}.
Finally, you should  read all parts of the series in  \href{https://students.linkedin.com/}{The Student Job Hunting Handbook} on LinkedIn and study all of the Tip Sheets. 

\section*{Social Media Marketing Plan}
%TO DO: add inforgraphics and a para on social media content from 2 and common mistakes listing
%1. https://moz.com/beginners-guide-to-social-media/linkedin
%2. https://www.accreditedschoolsonline.org/resources/managing-social-profiles-in-college/

A social media profile should be viewed as a brand. You need to create a brand that represent you in such a way that suggests you are a computer science professional who is a good hiring candidate or a good candidate for graduate schools. When you apply for jobs typically after your resume is checked, the majority of companies check the social profiles and do a Google search. You should view your personal brand and presence online the same as any company would.  Would a company want to do business with you?

 Managing your online reputation should be of interest to students looking for industry positions after graduation and  those considering graduate schools. Remember since nothing you share online is private, creating and maintaining a favorable online presence can make the difference between being accepted or rejected for a job or a graduate degree program.
The common mistakes that students make on social media include not using social media enough, posting inappropriate content, using poor writing, complaining about your job, colleagues, etc., trusting or not using proper privacy settings at all, among others. To clean up your social media google yourself, add Google Alerts to be notified when there is a new online content that includes your name, check your activity logs on social media and remove any inappropriate content, ensure that details on all social media platforms and your resume match, and do post relevant and interest content regularly.

\section*{LinkedIn Profile}
\label{linkedin}
LinkedIn has become the most popular social networking tool for professionals with  500+ million users. Are you on LinkedIn? The first step is
to create a profile page that is informative and concise. At the minimum, you should complete the following tasks to create and expand your LinkedIn account. 

\begin{enumerate}
	\item Include a professional photo of yourself.
	\item Create a unique headline.
	\item Create a professional summary, which should consist of one concise paragraph similar  to a summary in a cover letter.
	\item Add professional experience: included at least two jobs or volunteer positions.
	\item Add education, which should include: 
	\begin{itemize}
		\item  the correct school name that will enable you to connect with LinkedIn Alumni Network,
		\item anticipated degree with a graduation month/year.
	\end{itemize}	 
	\item Add the five skills and expertise keywords to the profile.
	\item Add location and industry to the profile.
	\item Create a unique, professional URL.
	\item Obtain at least one recommendation.
	\item Join at least one educational group and one professional group.
	\item Create at least five new connections, at least three connections have to be currently outside of Allegheny College (see the next section for more details). 
	\item Make an informative and a professional post (for example, sharing a relevant professional article). You should automate this task using a tool such as \href{https://buffer.com/}{buffer}. 
	\item Make a note of your LinkedIn statistics (number of views, number of connections) today and then again at the due time of the assignment. You will need this information in subsequent assignments.
\end{enumerate}

Additionally, you should review \href{https://university.linkedin.com/content/dam/university/global/en_US/site/pdf/LinkedIn%20Profile%20Checklist%20-%20College%20Students.pdf}{Profile CheckList for University Students} to ensure you have included other relevant information.  

\section*{Professional Networking}

To begin growing your professional network, you should first connect to your instructor by finding her profile and adapting the introduction from the standard  message to a more personal request. Then you should explore Allegheny College alumni in the field of computer science to see which companies they are employed in, what type of positions they hold, which geographical locations they are in, etc. Now, consider sending some of the alumni a connection request.  A blog on \href{https://blog.linkedin.com/2013/01/30/start-mapping-your-career-with-linkedin-alumni}{Mapping your Career with Alumni} maybe helpful for this task. Finally, you should create an account on \href{https://sites.allegheny.edu/gatorconnect/}{GatorConnect} to utilize it as another tool to connect to alumni. You need to set up your profile on GatorGrader according to specifications outlined in the \href{https://sites.allegheny.edu/gatorconnect/using-gator-connect/#updateprofile}{Using  Gator Connect} page.

\section*{Deliverables}
Your submission of this  assignment includes the completion of all of the tasks related to the LinkedIn profile, which are outlined in the bullet points in Section called ``LinkedIn Profile'' above. Your second deliverable is the creation and appropriate set up of your profile in Gator Connect.

\section*{Evaluation of Your  Assignment}

 Your grade for the assignment will be a function of the
whether or not it was submitted in a timely fashion and if your submission includes all of the required deliverables. In addition to evaluating the professional and informative structure of your LinkedIn and Gator Connect profiles, the instructor will also evaluate the number of connections you make, the relevance of the groups you join and the article you post.
 Please see the instructor if you have
questions about the evaluation of this  assignment.


\end{document}
